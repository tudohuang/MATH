\section{觀念與公式}

\subsection{數據的表示與整理}
數據分析從整理開始。\\
\textbf{方法}:
\begin{itemize}
    \item 頻率表、分組頻率表。
    \item 圖表:直方圖、折線圖、圓餅圖、莖葉圖、箱形圖(五數概要:$Q_1, Q_2, Q_3, \text{最小值}, \text{最大值}$)。
\end{itemize}
\textbf{應用}:視覺化數據。\\
\textbf{大學技巧}:用Python繪圖。

\subsection{中心量數}
衡量數據集中趨勢。\\
\textbf{公式}:
\begin{itemize}
    \item 平均數:$\bar{x} = \frac{\sum x_i}{n}$。
    \item 中位數:排序後中間值。
    \item 眾數:出現次數最多值。
\end{itemize}
\textbf{應用}:比較數據特性。\\
\textbf{大學技巧}:加權平均。

\subsection{離散程度}
衡量數據分散性。\\
\textbf{公式}:
\begin{itemize}
    \item 全距:最大值 - 最小值。
    \item 四分位距:$IQR = Q_3 - Q_1$。
    \item 變異數:$s^2 = \frac{\sum (x_i - \bar{x})^2}{n}$(樣本用$n-1$)。
    \item 標準差:$s = \sqrt{s^2}$。
\end{itemize}
\textbf{應用}:數據穩定性。\\
\textbf{大學技巧}:常態分佈68-95-99.7規則。

\subsection{數據分佈}
描述數據形狀。\\
\textbf{類型}:
\begin{itemize}
    \item 對稱分佈:平均數 $\approx$ 中位數。
    \item 偏態分佈:右偏(平均數 > 中位數)、左偏(平均數 < 中位數)。
    \item 常態分佈:鐘形曲線。
\end{itemize}
\textbf{應用}:判斷特性。\\
\textbf{大學技巧}:標準化$z = \frac{x - \mu}{\sigma}$。

\subsection{雙變量數據分析}
探索兩變量關係。\\
\textbf{方法}:
\begin{itemize}
    \item 散佈圖:觀察數據點分佈。
    \item 相關係數(皮爾森):$r = \frac{\sum (x_i - \bar{x})(y_i - \bar{y})}{\sqrt{\sum (x_i - \bar{x})^2 \sum (y_i - \bar{y})^2}}$,範圍$-1 \leq r \leq 1$。
    \item 迴歸直線:$y = ax + b$,斜率$a = \frac{\sum (x_i - \bar{x})(y_i - \bar{y})}{\sum (x_i - \bar{x})^2}$。
\end{itemize}
\textbf{性質}:
\begin{itemize}
    \item $r > 0$:正相關,$r < 0$:負相關,$r \approx 0$:無線性相關。
    \item $|r|$接近1表示強相關,接近0表示弱相關。
\end{itemize}
\textbf{應用}:預測與解釋。\\
\textbf{大學技巧}:最小平方法、$r^2$解釋變異。

\section{例題解析}

\subsection{例題1:數據整理}
10人身高(cm):155, 160, 165, 170, 155, 168, 172, 160, 165, 170。求五數概要。\\
\textbf{解}:排序:155, 155, 160, 160, 165, 165, 168, 170, 170, 172。\\
五數:最小155, $Q_1=157.5$, $Q_2=165$, $Q_3=170$, 最大172。

\subsection{例題2:中心與離散}
數據:4, 6, 8, 10, 12。求平均數與標準差。\\
\textbf{解}:$\bar{x} = \frac{40}{5} = 8$。\\
$s^2 = \frac{(4-8)^2 + (6-8)^2 + (8-8)^2 + (10-8)^2 + (12-8)^2}{5} = 8$,$s = \sqrt{8} \approx 2.83$。

\subsection{例題3:相關係數}
身高(cm)與體重(kg):$(160, 50), (165, 55), (170, 60), (175, 65)$。求$r$。\\
\textbf{解}:$\bar{x} = 167.5$, $\bar{y} = 57.5$。\\
計算:$\sum (x_i - \bar{x})(y_i - \bar{y}) = 150$, $\sum (x_i - \bar{x})^2 = 125$, $\sum (y_i - \bar{y})^2 = 125$。\\
$r = \frac{150}{\sqrt{125 \cdot 125}} = 1$,完全正相關。\\
\textbf{大學技巧}:驗證$r^2 = 1$。

\subsection{例題4:迴歸直線}
同上數據,求迴歸直線。\\
\textbf{解}:斜率$a = \frac{150}{125} = 1.2$,截距$b = \bar{y} - a \bar{x} = 57.5 - 1.2 \cdot 167.5 = -143.5$。\\
方程:$y = 1.2x - 143.5$。

\subsection{例題5:應用題}
手機時數:1, 2, 3, 5, 9。求IQR與離群值。\\
\textbf{解}:$Q_1 = 1.5$, $Q_2 = 3$, $Q_3 = 7$,$IQR = 5.5$。\\
範圍:$-6.75$ 至 $15.25$,無離群值。

\section{圖形展示}
散佈圖(例題3數據):
\begin{tikzpicture}
    \begin{axis}[
        xlabel=身高 (cm),
        ylabel=體重 (kg),
        xmin=155, xmax=180,
        ymin=45, ymax=70,
        grid=both,
        width=8cm, height=6cm
    ]
    \addplot[mark=*, blue] coordinates {(160,50) (165,55) (170,60) (175,65)};
    \end{axis}
\end{tikzpicture}

\section{題庫}
\begin{enumerate}[label=\arabic*.]
    % 計算題 (10)
    \item 數據:3, 5, 7, 9。求平均數。
    \item 數據:2, 4, 6, 8, 10。求中位數。
    \item 數據:1, 1, 2, 3, 4。求眾數。
    \item 數據:5, 10, 15。求全距。
    \item 數據:2, 4, 6。求標準差。
    \item 數據:1, 3, 5, 7, 9。求$Q_2$。
    \item 數據:4, 8, 12, 16。求變異數。
    \item 數據:10, 20, 30, 40。求IQR。
    \item 數據:2, 2, 3, 3, 4。求平均數。
    \item 數據:5, 6, 7, 8。求標準差。
    % 應用題 (20)
    \item 身高(cm):150, 155, 160, 165, 170。求五數概要。
    \item 銷量:10, 12, 15, 20, 25。求標準差。
    \item 分數:60, 70, 80, 90, 100。求離群值。
    \item 身高與體重:$(150, 45), (160, 50), (170, 55)$。求$r$。
    \item 手機時數:1, 2, 3, 5, 9。求IQR。
    \item 溫度(℃):20, 22, 25, 28, 30。求平均數與全距。
    \item 數據:5, 5, 6, 7, 8, 9。判斷偏態。
    \item 銷售額(萬):2, 4, 6, 8, 10。求$r$與時間(1-5)。
    \item 身高:145, 150, 155, 160, 165, 170。求$Q_1, Q_3$。
    \item 分數:50, 60, 70, 80, 90。求中位數與離群值。
    \item 步數:3000, 4000, 5000, 6000, 7000。求迴歸斜率。
    \item 價格與銷量:$(10, 50), (20, 40), (30, 30)$。求$r$。
    \item 成績:40, 50, 60, 70, 80, 90。求五數概要。
    \item 降雨量(mm):0, 2, 5, 10, 20。求標準差。
    \item 年齡:20, 25, 30, 35, 40, 45。求平均數與IQR。
    \item 身高與鞋碼:$(160, 38), (165, 39), (170, 40)$。求迴歸直線。
    \item 顧客數:5, 10, 15, 20, 25。求變異數。
    \item 用電(度):50, 60, 70, 80, 90。求全距與中位數。
    \item 數據:1, 3, 5, 7, 9, 11。求$Q_2$與莖葉圖。
    \item 身高與體重:$(155, 48), (160, 52), (165, 56), (170, 60)$。求$r$。
    % 觀念題 (10)
    \item 平均數與中位數差異?
    \item 標準差如何衡量分散?
    \item 何謂五數概要?
    \item $r$的範圍與意義?
    \item 何謂右偏分佈?
    \item 全距與IQR區別?
    \item 變異數計算步驟?
    \item 散佈圖如何判斷相關性?
    \item 何謂離群值?
    \item 常態分佈特徵?
    % 進階題 (10)
    \item 數據:2, 4, 6, 8, 10, 12。求$x=10$的$z$-分數。
    \item 數據:$\bar{x}=50, s=3$,求$x=56$的$z$-分數。
    \item 數據:1, 2, 3, 4, 5。求樣本變異數。
    \item 身高與體重:$(150, 45), (160, 50), (170, 55)$。求迴歸直線。
    \item 數據:10, 20, 30, 40, 50。求$Q_3$與離群值範圍。
    \item 常態分佈:$\mu=100, \sigma=15$,求$P(85 < x < 115)$。
    \item 銷量:5, 10, 15, 20(天數1-4)。求$r$。
    \item 數據:3, 6, 9, 12, 15。求樣本標準差。
    \item 年齡:20, 25, 30, 35, 40。求$r$與時間(1-5)。
    \item 身高與時間:$(1, 150), (2, 155), (3, 160)$。求迴歸方程。
    % 挑戰題 (10)
    \item 數據:1, 3, 5, 7, 9。求樣本與母體標準差。
    \item 身高:150, 155, 160, 165, 170, 200。求離群值。
    \item 常態分佈:$\mu=50, \sigma=10$,求$x=70$百分位。
    \item 數據:2, 4, 6, 8, 10。求$r$與時間(1-5)。
    \item 身高與體重:$(160, 50), (165, 55), (170, 60), (175, 65)$。求$r$與迴歸。
    \item 溫度:20, 22, 25, 28, 30, 40。求IQR與離群值。
    \item 數據:10, 15, 20, 25, 30。求樣本變異數與$r$。
    \item 身高與成績:$(150, 60), (160, 70), (170, 80)$。求迴歸直線。
    \item 數據:5, 10, 15, 20, 25, 30。求五數與常態假設。
    \item 調查:1, 2, 3, 5, 10, 15。求$r$與時間(1-6)。
\end{enumerate}