
\section{觀念與公式}

\subsection{基本概念}
排列與組合處理選擇與排列問題。\\
\textbf{定義}:
\begin{itemize}
    \item 排列:$P(n, k) = \frac{n!}{(n-k)!}$,從n個取k個有序排列。
    \item 組合:$\binom{n}{k} = \frac{n!}{k!(n-k)!}$,從n個取k個無序組合。
    \item 階乘:$n! = n \cdot (n-1) \cdots 1$,$0! = 1$。
\end{itemize}
\textbf{應用}:計數問題。\\
\textbf{大學技巧}:用生成函數求解。

\subsection{排列的應用}
\textbf{公式}:
\begin{itemize}
    \item 全排列:$n!$。
    \item 環狀排列:$(n-1)!$。
    \item 有重複排列:$\frac{n!}{n_1! n_2! \cdots n_k!}$。
    \item \textbf{Hn取k}:從n個不同物取k個(可重複),$n^k$。
\end{itemize}
\textbf{應用}:排隊、密碼。\\
\textbf{大學技巧}:容斥原理處理限制。

\subsection{組合的應用}
\textbf{公式}:
\begin{itemize}
    \item 基本組合:$\binom{n}{k}$。
    \item 二項式定理:$(a + b)^n = \sum_{k=0}^n \binom{n}{k} a^{n-k} b^k$。
    \item 性質:$\binom{n}{k} = \binom{n}{n-k}$,帕斯卡公式$\binom{n}{k} + \binom{n}{k-1} = \binom{n+1}{k}$。
\end{itemize}
\textbf{應用}:選物、分組。\\
\textbf{大學技巧}:組合恆等式。

\subsection{卡特蘭數}
定義:$C_n = \frac{1}{n+1} \binom{2n}{n}$。\\
\textbf{應用}:
\begin{itemize}
    \item 括號配對:$C_3 = 5$。
    \item 不越界路徑:從$(0,0)$到$(n,n)$。
\end{itemize}
\textbf{值}:$C_0 = 1, C_1 = 1, C_2 = 2, C_3 = 5, C_4 = 14$。\\
\textbf{大學技巧}:遞推$C_n = \sum_{i=0}^{n-1} C_i C_{n-1-i}$。

\section{例題解析}

\subsection{例題1:條件排列}
5人排隊,A, B相鄰,求方法數。\\
\textbf{解}:視A, B為一組,則4個單位(AB, 3人),全排列$4! = 24$,AB內部2種排法,總數$24 \cdot 2 = 48$。\\
\textbf{大學技巧}:容斥法驗證。

\subsection{例題2:分組應用}
從6人選3人分給A組,餘下給B組,求方法數。\\
\textbf{解}:選3人給A組,$\binom{6}{3} = 20$,餘下自動分B組,總數$20$。\\
\textbf{大學技巧}:考慮組內排列。

\subsection{例題3:卡特蘭數應用}
求4個節點的二叉樹數量。\\
\textbf{解}:二叉樹數為$C_4 = \frac{1}{5} \binom{8}{4} = \frac{70}{5} = 14$。\\
\textbf{大學技巧}:用路徑計數。

\subsection{例題4:重複排列應用}
從3種顏色選4個塗4格(可重複),求方法數。\\
\textbf{解}:$H(3, 4) = 3^4 = 81$。\\
\textbf{大學技巧}:生成函數$(x_1 + x_2 + x_3)^4$。

\subsection{例題5:二項式應用}
求$(3x - 2)^5$的$x^3$項係數。\\
\textbf{解}:$(3x)^{5-k} (-2)^k$,$5-k=3$,$k=2$:
\[
\binom{5}{2} \cdot 3^3 \cdot (-2)^2 = 10 \cdot 27 \cdot 4 = 1080
\]
\textbf{大學技巧}:用係數提取。



\section{題庫}
\begin{enumerate}[label=\arabic*.]
    % 計算題 (10)
    \item 求$P(5, 3)$。
    \item 求$\binom{6}{2}$。
    \item 求$4!$。
    \item 用A, A, B排3位序列,求方法數。
    \item 求$(x + 2)^4$的$x^2$項係數。
    \item 求$P(7, 2)$。
    \item 求$\binom{8}{3}$。
    \item 用1, 1, 2排3位序列,求方法數。
    \item 從4人排成一圈,求方法數。
    \item 求$\binom{5}{4}$。
    % 應用題 (20)
    \item 5人排隊,A, B相鄰,求方法數。
    \item 從6人選3人組成委員會,A必選,求方法數。
    \item 4人圍圓桌,A, B不相鄰,求方法數。
    \item 從8本書選4本分給2人,每人2本,求方法數。
    \item 用0, 1組成5位數(可重複),至少1個0,求方法數。
    \item 6個不同球分3堆(可空),求方法數。
    \item 從5男3女選4人,至少1女,求方法數。
    \item 求$(2x - 1)^6$的$x^4$項係數。
    \item 5人排隊,A在B前,求方法數。
    \item 從7個數字選3個,至少1奇數,求方法數。
    \item 6人圍圓桌,A, B, C順時針連續,求方法數。
    \item 用A, B, C排4位序列,至少2個A,求方法數。
    \item 從10人選5人,至少2人為特定組(4人),求方法數。
    \item 5個不同物分2堆(不可空),求方法數。
    \item 從4種顏色選3格塗色(可重複),求方法數。
    \item 7人排隊,A, B相鄰且C在D前,求方法數。
    \item 求$C_4$(卡特蘭數)。
    \item 從6個數字選3個排成3位數,至少1個5,求方法數。
    \item 8人選4人分2組,每組2人,求方法數。
    \item 從5人選3人排隊,A, B不相鄰,求方法數。
    % 觀念題 (10)
    \item 證$\binom{n}{k} = \binom{n}{n-k}$。
    \item 說明環狀排列的公式推導。
    \item 證帕斯卡恆等式。
    \item 若有重複物,排列公式為何?
    \item 說明二項式定理如何求係數。
    \item 卡特蘭數如何應用於路徑計數?
    \item 若從n個取k個可重複,方法數為何?
    \item 證$P(n, k) = n \cdot (n-1) \cdots (n-k+1)$。
    \item 說明組合與排列的區別。
    \item 二項式展開的總係數和為何?
    % 進階題 (10)
    \item 從6人選3人排隊,A必須在首位,求方法數。
    \item 求$(3x + 2)^5$的$x^3$項係數。
    \item 用A, A, B, C, D排5位序列,求方法數。
    \item 求$C_5$(卡特蘭數)。
    \item 從8人選4人,A, B必選,求方法數。
    \item 若$(x - 3)^7$的$x^5$項為$kx^5$,求$k$。
    \item 6人排隊,A, B, C中至少2人相鄰,求方法數。
    \item 從5種數字選4個(可重複),至少1個1,求方法數。
    \item 7人圍圓桌,A, B相鄰且C不在D旁,求方法數。
    \item 從9人選5人,至少3人為特定組(4人),求方法數。
    % 挑戰題 (10)
    \item 6人排隊,A, B相鄰且C, D不相鄰,求方法數。
    \item 求$(2x - y)^6$的$x^4 y^2$項係數。
    \item 5個不同球分3堆,至少1堆空,求方法數。
    \item 用A, A, A, B, B, C排6位序列,求方法數。
    \item 從7男3女選5人,至少2男2女,求方法數。
    \item 8人圍圓桌,A, B, C順時針連續且D不在E旁,求方法數。
    \item 若$(x + 2y)^8$的$x^4 y^4$項為$kx^4 y^4$,求$k$。
    \item 從6個數字選3個,至少1個5且1個7,求方法數。
    \item 5人排隊,A在B前且C在D前,求方法數。
    \item 從10本書選5本分3人,每人至少1本,求方法數。
\end{enumerate}

