% chapter1.tex

\section{觀念與公式}

\subsection{數系}
數系是數學的基石,從最簡單的計數開始,逐步擴展到抽象的數學對象。以下逐一介紹:
\begin{itemize}
    \item \textbf{自然數} ($\mathbb{N}$):$1, 2, 3, \dots$,用於計數,最基本的數集合。
    \item \textbf{整數} ($\mathbb{Z}$):$\dots, -2, -1, 0, 1, 2, \dots$,加入負數解決減法問題,如$3 - 5 = -2$。
    \item \textbf{有理數} ($\mathbb{Q}$):形如$\frac{p}{q}$($p, q \in \mathbb{Z}, q \neq 0$),解決除法需求,例如$\frac{1}{2}$。
    \item \textbf{無理數}:無法寫成分數的實數,如$\sqrt{2}$(證明:假設$\sqrt{2} = \frac{p}{q}$,則$2q^2 = p^2$,$p$必為偶數,矛盾),還有$\pi, e$。
    \item \textbf{實數} ($\mathbb{R}$):包含有理數與無理數,形成連續的數線,滿足完備性(每個有上界的非空集合有最小上界)。
    \item \textbf{複數} ($\mathbb{C}$):形如$a + bi$($a, b \in \mathbb{R}, i = \sqrt{-1}$),解決$x^2 + 1 = 0$等方程。
\end{itemize}
\textbf{應用}:實數用於測量(如長度),複數用於電路分析與振動模型。\\
\textbf{大學技巧}:複數的極形式$r(\cos\theta + i\sin\theta)$,其中$r = \sqrt{a^2 + b^2}$,$\theta = \tan^{-1}\left(\frac{b}{a}\right)$。例如,$3 + 4i$的模為$5$,幅角約$53.13^\circ$,乘除時模相乘、幅角相加減,比代數形式更快。

\subsection{絕對值運算}
絕對值表示數到原點的距離,定義為:
\[
|x| =
\begin{cases} 
x, & \text{若 } x \geq 0, \\
-x, & \text{若 } x < 0.
\end{cases}
\]
例如,$|3| = 3$,$|-3| = 3$。\\
\textbf{性質}:
\begin{enumerate}
    \item $|x| \geq 0$:距離非負。
    \item $|-x| = |x|$:對稱性。
    \item $|x \cdot y| = |x| \cdot |y|$:乘法分配。
    \item 三角不等式:$|x + y| \leq |x| + |y|$,證明:考慮平方$(x + y)^2 \leq (|x| + |y|)^2$。
\end{enumerate}
\textbf{應用}:測量距離(如$|x - a|$是$x$到$a$的距離),解決含絕對值的方程。\\
\textbf{大學技巧}:解$|x - a| < b$時,直接轉為$a - b < x < a + b$。例如,$|x - 2| < 3$即$-1 < x < 5$,幾何上是以$2$為中心,半徑$3$的區間。

\subsection{算術不等式}
不等式是優化與比較的工具,常見如下:
\begin{itemize}
    \item \textbf{AM-GM不等式}:對於非負數$a, b$,$\frac{a + b}{2} \geq \sqrt{ab}$。證明:
    \begin{itemize}
        \item \textbf{代數法}:$(a - b)^2 \geq 0$,展開得$a^2 - 2ab + b^2 \geq 0$,整理為$(a + b)^2 \geq 4ab$,開根號即得。等號當$a = b$。
        \item \textbf{拉格朗日乘子法}:求$ab$最大值,受限於$a + b = k$。定義函數$f(a, b) = ab + \lambda (a + b - k)$,求偏導:
        \[
        \frac{\partial f}{\partial a} = b + \lambda = 0, \quad \frac{\partial f}{\partial b} = a + \lambda = 0, \quad a + b = k
        \]
        得$a = b$,代入$a + b = k$,$a = b = \frac{k}{2}$,$ab = \left(\frac{k}{2}\right)^2$,驗證為最大值。
    \end{itemize}
    \item \textbf{柯西-施瓦茨不等式}:$(a_1b_1 + a_2b_2)^2 \leq (a_1^2 + a_2^2)(b_1^2 + b_2^2)$,幾何上是向量內積與模的關係。
\end{itemize}
\textbf{應用}:求極值,如$a + b = 10$時,$ab$最大值為25。\\
\textbf{大學技巧}:用HM(調和平均)與GM、AM串聯:$\frac{2}{\frac{1}{a} + \frac{1}{b}} \leq \sqrt{ab} \leq \frac{a + b}{2}$,解決複雜優化問題。拉格朗日法適用於多變量受限極值。

\subsection{根號運算}
根號是平方根的符號,$\sqrt{x}$表示滿足$y^2 = x$且$y \geq 0$的數。\\
\textbf{基本運算}:
\begin{itemize}
    \item $\sqrt{ab} = \sqrt{a} \cdot \sqrt{b}$($a, b \geq 0$),如$\sqrt{12} = \sqrt{4 \cdot 3} = 2\sqrt{3}$。
    \item $\sqrt{\frac{a}{b}} = \frac{\sqrt{a}}{\sqrt{b}}$($a \geq 0, b > 0$),如$\sqrt{\frac{9}{4}} = \frac{3}{2}$。
    \item 化簡:提取完全平方數,如$\sqrt{50} = \sqrt{25 \cdot 2} = 5\sqrt{2}$。
\end{itemize}
\textbf{複數根號}:對於$a + bi$,求$\sqrt{a + bi}$,設結果為$x + yi$:
\[
(x + yi)^2 = a + bi \implies x^2 - y^2 + 2xyi = a + bi
\]
比較實虛部:
\[
x^2 - y^2 = a, \quad 2xy = b, \quad |x + yi| = \sqrt{a^2 + b^2}
\]
例如,$\sqrt{3 + 4i}$,模為$\sqrt{25} = 5$,解得$x = 2, y = 1$(正根)。\\
\textbf{應用}:化簡表達式,解決二次方程。\\
\textbf{大學技巧}:複數根號用極形式更快,$\sqrt{re^{i\theta}} = \sqrt{r}e^{i\frac{\theta}{2}}$。

\section{例題解析}

\subsection{例題1:複數運算(計算題)}
計算$(3 + 4i)(2 - i)$。\\
\textbf{解}:展開:
\[
(3 + 4i)(2 - i) = 6 - 3i + 8i - 4i^2 = 6 + 5i + 4 = 10 + 5i
\]
\textbf{大學技巧}:極形式下,$3 + 4i$模為$5$,$2 - i$模為$\sqrt{5}$,幅角相加後轉回,但此題代數法更直接。

\subsection{例題2:絕對值不等式(應用題)}
解$|2x - 3| < 5$。\\
\textbf{解}:去絕對值:
\[
-5 < 2x - 3 < 5 \implies -2 < 2x < 8 \implies -1 < x < 4
\]
解為$x \in (-1, 4)$。\\
\textbf{大學技巧}:幾何法,$2x - 3$到0的距離小於5,$x$到$\frac{3}{2}$的距離小於$\frac{5}{2}$。

\subsection{例題3:AM-GM應用(觀念題)}
若$a + b = 10$,求$ab$最大值。\\
\textbf{解}:由AM-GM:
\[
\frac{a + b}{2} \geq \sqrt{ab} \implies 5 \geq \sqrt{ab} \implies ab \leq 25
\]
當$a = b = 5$時,$ab = 25$,故最大值為25。\\
\textbf{大學技巧}:拉格朗日法驗證,$f(a) = a(10 - a)$,導數為0時$a = 5$。

\subsection{例題4:根號化簡(計算題)}
化簡$\sqrt{48} + \sqrt{75}$。\\
\textbf{解}:
\[
\sqrt{48} = \sqrt{16 \cdot 3} = 4\sqrt{3}, \quad \sqrt{75} = \sqrt{25 \cdot 3} = 5\sqrt{3}
\]
\[
\sqrt{48} + \sqrt{75} = 4\sqrt{3} + 5\sqrt{3} = 9\sqrt{3}
\]
\textbf{大學技巧}:檢查是否有進一步因式分解,此題已最簡。

\section{圖形展示}
複數$3 + 4i$在複數平面:
\begin{tikzpicture}
    \begin{axis}[
        axis lines=middle,
        xlabel=實部,
        ylabel=虛部,
        xmin=-1, xmax=5,
        ymin=-1, ymax=5,
        grid=both,
        width=6cm, height=6cm
    ]
    \addplot[mark=*,red] coordinates {(3,4)} node[above right] {$3 + 4i$};
    \addplot[dashed] coordinates {(0,0) (3,4)};
    \end{axis}
\end{tikzpicture}

\section{題庫}
\begin{enumerate}[label=\arabic*.]
    % 計算題 (10)
    \item 計算$(2 + 3i) + (4 - 5i)$。
    \item 求$|5 - 2i|$。
    \item 解$|x - 2| = 3$。
    \item 計算$\frac{1 + i}{1 - i}$。
    \item 求$(1 + i)^3$。
    \item 解$|2x + 1| = 5$。
    \item 計算$|-3 + 4i|^2$。
    \item 化簡$\sqrt{18} + \sqrt{50}$。
    \item 解$|x + 3| = |x - 1|$。
    \item 計算$\sqrt{5 + 12i}$的模。
    % 應用題 (10)
    \item 若$|x - 5| < 2$,求$x$範圍。
    \item 若$a + b = 12$,求$ab$最大值。
    \item 一點$x$到起點距離為$|x|$,到終點10的距離為何?
    \item 若$a, b > 0$,$a + b = 8$,求$\frac{1}{a} + \frac{1}{b}$最小值。
    \item 價格$x$滿足$|x - 100| \leq 10$,求範圍。
    \item 若$x^2 + y^2 = 2$,求$|x| + |y|$最大值。
    \item 圓心$(3, 0)$,半徑2,點$(x, 0)$在圓內,求$x$範圍。
    \item 若$a + b + c = 15$,求$abc$最大值。
    \item 若$|x - 3| < |x + 2|$,求$x$範圍。
    \item 若$a^2 + b^2 = 4$,求$a + b$最大值。
    % 觀念題 (10)
    \item 證明$|x + y| \leq |x| + |y|$。
    \item 若$|x| = 0$,$x$為何?
    \item AM-GM等號條件是什麼?
    \item 複數$z$與$\overline{z}$的模相等嗎?
    \item 解釋$|x - a| < b$的幾何意義。
    \item 證明$|x - y| \geq ||x| - |y||$。
    \item 若$a, b > 0$,比較$\frac{a + b}{2}$與$\sqrt{ab}$。
    \item $z$與$-z$的模是否相等?
    \item 若$|x| = |y|$,$x = y$嗎?
    \item 證明$\sqrt{ab} \leq \frac{a + b}{2}$($a, b \geq 0$)。
    % 進階題 (10)
    \item 解$|x - 1| + |x + 1| = 4$。
    \item 若$|z - 1| = 2$,描述$z$軌跡。
    \item 證明$\frac{a + b + c}{3} \geq \sqrt[3]{abc}$。
    \item 解$|x - 2| + |x - 3| = 1$。
    \item 求$\max\{|x| + |y| \mid x^2 + y^2 = 1\}$。
    \item 若$|z| = 1$,求$z + \frac{1}{z}$實部。
    \item 解$|x - 1| < |x + 2|$。
    \item 若$a + b = 10$,求$a^2 + b^2$最小值。
    \item 化簡$\sqrt{12} - \sqrt{27} + \sqrt{75}$。
    \item 若$|z - i| = |z + i|$,求$z$性質。
    % 挑戰題 (20)
    \item 解$|x - 1| + |x - 2| + |x - 3| = 5$。
    \item 若$a + b = 6$,求$(a - 3)^2 + (b - 3)^2$最小值。
    \item 證明$\frac{a}{b} + \frac{b}{a} \geq 2$,$a, b > 0$。
    \item 求所有$z$滿足$|z - 1| = |z - i|$。
    \item 若$x + y + z = 3$,求$x^2 + y^2 + z^2$最小值。
    \item 解$||x - 1| - 2| = 3$。
    \item 若$|z - 2| = 1$且$|z - i| = 2$,求$z$。
    \item 證明$a^2 + b^2 \geq 4ab$,並找出等號條件。
    \item 解$|x - 1| + |x + 1| = |x - 2| + |x + 2|$。
    \item 若$a, b, c > 0$,$a + b + c = 3$,求$\frac{1}{a} + \frac{1}{b} + \frac{1}{c}$最小值。
    \item 求滿足$|x - 1| + |x - 2| = 1$的$x$個數。
    \item 若$x + y = xy$,求$x^2 + y^2$最小值。
    \item 證明$\frac{a + b}{2} \geq \sqrt{ab} \geq \frac{2}{\frac{1}{a} + \frac{1}{b}}$。
    \item 解$|x| + |x - 2| + |x - 4| = 6$。
    \item 若$|z - 1| + |z + 1| = 4$,求$z$軌跡。
    \item 求$\max\{x + y \mid x^2 + y^2 = 1\}$。
    \item 若$a + b + c = 9$,求$a^2 + b^2 + c^2$最小值。
    \item 解$|x - 1| = 2|x + 1|$。
    \item 若$|z| = 2$,求$|z - 1| + |z + 1|$最大值。
    \item 計算$\sqrt{8 + 6i}$。
\end{enumerate}

