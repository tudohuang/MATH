\section{觀念與公式}

\subsection{基本概念}
機率衡量事件可能性。\\
\textbf{定義}:
\begin{itemize}
    \item 樣本空間:所有可能結果。
    \item 事件:樣本空間子集。
    \item 機率:$P(A) = \frac{\text{有利結果數}}{\text{總結果數}}$(古典機率)。
\end{itemize}
\textbf{應用}:簡單計數。\\
\textbf{大學技巧}:集合論描述。

\subsection{機率的性質與運算}
\textbf{公式}:
\begin{itemize}
    \item 補事件:$P(\overline{A}) = 1 - P(A)$。
    \item 互斥事件:$P(A \cup B) = P(A) + P(B)$。
    \item 獨立事件:$P(A \cap B) = P(A) \cdot P(B)$。
\end{itemize}
\textbf{應用}:多事件組合。\\
\textbf{大學技巧}:樹狀圖分析。

\subsection{條件機率}
定義:$P(A|B) = \frac{P(A \cap B)}{P(B)}$,$P(B) > 0$。\\
\textbf{性質}:$P(A \cap B) = P(A|B) \cdot P(B)$。\\
\textbf{應用}:依序事件。\\
\textbf{大學技巧}:貝氏定理$P(A|B) = \frac{P(B|A) \cdot P(A)}{P(B)}$。

\subsection{排列組合與機率}
用排列組合計算機率。\\
\textbf{應用}:
\begin{itemize}
    \item 樣本空間:$n!$、$P(n,k)$、$\binom{n}{k}$。
    \item 有利事件:條件計數。
\end{itemize}
\textbf{大學技巧}:生成函數。

\subsection{期望值}
衡量隨機變量的平均結果。\\
\textbf{公式}:$E(X) = \sum x_i \cdot P(x_i)$,$x_i$為結果,$P(x_i)$為機率。\\
\textbf{應用}:遊戲收益、試驗平均值。\\
\textbf{大學技巧}:二項分佈期望值$E(X) = np$。

\section{例題解析}

\subsection{例題1:基本機率}
一袋有3紅4藍球,隨機抽1球,求紅球機率。\\
\textbf{解}:$P(\text{紅}) = \frac{3}{7}$。

\subsection{例題2:互斥事件}
擲骰子,求點數為奇數或6的機率。\\
\textbf{解}:$P(\text{奇}) = \frac{3}{6}$,$P(6) = \frac{1}{6}$,$P(\text{奇或6}) = \frac{3}{6} + \frac{1}{6} = \frac{2}{3}$。

\subsection{例題3:條件機率}
袋有5紅3藍,抽2次(放回),求第2紅且第1藍機率。\\
\textbf{解}:$P(\text{第1藍}) = \frac{3}{8}$,$P(\text{第2紅}) = \frac{5}{8}$,$P = \frac{15}{64}$。

\subsection{例題4:排列組合應用}
5人選3人排隊,A在B前機率。\\
\textbf{解}:總數$P(5,3) = 60$,A在B前$24$,$P = \frac{2}{5}$。

\subsection{例題5:期望值}
擲骰子,贏點數美元,求期望收益。\\
\textbf{解}:$E(X) = 1 \cdot \frac{1}{6} + 2 \cdot \frac{1}{6} + 3 \cdot \frac{1}{6} + 4 \cdot \frac{1}{6} + 5 \cdot \frac{1}{6} + 6 \cdot \frac{1}{6} = \frac{21}{6} = 3.5$美元。

\section{圖形展示}
樹狀圖(例題3):
\begin{tikzpicture}
    \node (A) at (0,0) {開始};
    \node (B) at (2,1) {藍 $\frac{3}{8}$};
    \node (C) at (2,-1) {紅 $\frac{5}{8}$};
    \node (D) at (4,2) {藍 $\frac{3}{8}$};
    \node (E) at (4,0) {紅 $\frac{5}{8}$};
    \draw (A) -- (B) -- (D);
    \draw (A) -- (B) -- (E);
    \draw (A) -- (C);
\end{tikzpicture}

\section{題庫}
\begin{enumerate}[label=\arabic*.]
    % 計算題 (10)
    \item 擲骰子,求點數4機率。
    \item 袋3紅2藍,求紅機率。
    \item 擲2骰子,求和7機率。
    \item 抽撲克,求紅心機率。
    \item 擲2幣,求至少1正機率。
    \item 袋4白3黑,求白機率。
    \item 擲骰子,求點數<4機率。
    \item 抽2牌(放回),求皆黑機率。
    \item 擲骰子,求期望點數。
    \item 袋5球3紅,求非紅機率。
    % 應用題 (20)
    \item 袋3紅4藍,抽2次(放回),求皆紅機率。
    \item 5人排隊,A, B相鄰機率。
    \item 擲2骰子,求和偶數機率。
    \item 袋5紅3藍,抽2次(不放回),求第2紅且第1藍機率。
    \item 抽撲克,求第1A且第2K機率(不放回)。
    \item 6人圍圓桌,A, B不相鄰機率。
    \item 袋4白6黑,抽2次(不放回),求皆白機率。
    \item 擲3幣,求恰2正機率。
    \item 遊戲:贏5元($\frac{1}{3}$),輸2元($\frac{2}{3}$),求期望收益。
    \item 袋3紅2藍1綠,抽2次(放回),求不同色機率。
    \item 擲骰子2次,求第2>第1機率。
    \item 5張牌選2張,求皆紅心機率。
    \item 袋6球4紅,抽2次(不放回),求至少1紅機率。
    \item 4男3女選2人,至少1女機率。
    \item 擲骰子,點數奇數得3元,偶數失2元,求期望收益。
    \item 袋5白3黑,抽2次(不放回),求第2白|第1黑機率。
    \item 6人排隊,A在B前機率。
    \item 抽3牌(放回),求至少1黑機率。
    \item 袋4紅3藍,抽2次(放回),求恰1紅期望次數。
    \item 擲2骰子,求和>10機率。
    % 觀念題 (10)
    \item 何謂樣本空間?
    \item 互斥事件機率如何算?
    \item 獨立事件定義?
    \item 條件機率公式?
    \item 補事件意義?
    \item 古典與實驗機率差異?
    \item 如何用排列計算機率?
    \item 何謂期望值?
    \item 機率值範圍?
    \item 樹狀圖如何幫機率計算?
    % 進階題 (10)
    \item 袋3紅4藍,抽2次(不放回),求第2紅|第1紅機率。
    \item 擲2骰子,求和8|第1為4機率。
    \item 5人排隊,A, B, C順序機率。
    \item 袋5白3黑,抽3次(放回),求恰2白機率。
    \item 抽撲克,求第2紅|第1黑機率(不放回)。
    \item 擲3幣,求期望正面次數。
    \item 袋4紅3藍,抽2次(不放回),求不同色機率。
    \item 從7人選4人,A, B必選機率。
    \item 擲2骰子,求和<6|差<3機率。
    \item 袋5球3紅,抽2次(放回),求紅次數期望值。
    % 挑戰題 (10)
    \item 袋3紅2藍,抽3次(不放回),求至少2紅機率。
    \item 5人排隊,A, B相鄰且C在D前機率。
    \item 擲2骰子,求和9|至少1個5機率。
    \item 袋4白5黑,抽3次(放回),求至少2白機率。
    \item 遊戲:贏10元($\frac{1}{4}$),贏5元($\frac{1}{2}$),輸3元($\frac{1}{4}$),求期望收益。
    \item 抽撲克3次(不放回),求恰1A機率。
    \item 從8人選3人,至少2男(5男3女)機率。
    \item 擲3骰子,求和10機率。
    \item 袋3紅4藍,抽2次(不放回),求第2紅|第1藍機率。
    \item 擲2骰子,求和期望值。
\end{enumerate}